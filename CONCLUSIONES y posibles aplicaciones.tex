\section{Conclusiones y posibles aplicaciones}

En este trabajo se ha llevado a cabo el diseño de una línea de transmisión bifilar, cumpliendo con las especificaciones establecidas para impedancia característica, capacitancia, atenuación y velocidad de propagación. Se ha obtenido un diseño óptimo que minimiza las pérdidas y maximiza la eficiencia de la transmisión.

Una de las aplicaciones más prometedoras de esta línea de transmisión bifilar es en el campo de las antenas dipolo. Al utilizar esta línea como alimentador de una antena dipolo, se puede lograr una mejor adaptación de impedancias, reduciendo las pérdidas por reflexión y mejorando la eficiencia de radiación. Además, la baja atenuación de la línea permite transmitir señales a mayores distancias sin una degradación significativa de la señal.

El diseño de una línea de transmisión bifilar es un tema de gran relevancia en el campo de las telecomunicaciones. Los resultados obtenidos en este trabajo demuestran la viabilidad de diseñar líneas de transmisión personalizadas para aplicaciones específicas, como las antenas dipolo, antenas de radio, TV, etc.



\pagebreak
