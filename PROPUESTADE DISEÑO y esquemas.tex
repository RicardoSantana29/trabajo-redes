\section{Propuesta de diseño y esquemas}

Basándose en el libro de \textit{Teoría Electromagnética} para lienas bifilares:

Bajas frecuencias:

\begin{align}
    L = \frac{\mu}{\pi}\left( \frac{1}{4} + arccosh \left( \frac{d}{2a} \right) \right)
\end{align}

\begin{align}
    R = \frac{2}{\sigma_c \pi a^2}
\end{align}

\begin{align}
    C = \frac{\pi \epsilon}{  arccosh \left( \frac{d}{2a} \right) }
\end{align}

\begin{align}
    G = \frac{\pi \sigma_d}{cosh^{-1} \left( \frac{d}{2a} \right)}
\end{align}

Altas frecuencias:

\begin{equation} \label{eq:L}
        \begin{aligned}
        L =    \frac{\mu}{\pi}    arccosh \left( \frac{d}{2a} \right) \approx \frac{\mu }{ \pi } ln \left( \frac{d}{a} \right) \text{;} &\quad    a \ll d
        \end{aligned}
        \end{equation}

\begin{equation} \label{eq:C}
        \begin{aligned}
        C =  \frac{  \pi \epsilon  }{   arccosh \left( \frac{d}{2a} \right) } \approx \frac{\pi \epsilon}{ ln( \frac{d}{a} ) }    \text{;} &\quad    a \ll d
        \end{aligned}
        \end{equation}

\begin{align} \label{eq:R}
    R = \frac{1}{\pi a l \sigma_c}
\end{align}

\begin{comment}
\begin{align}
    Z_o = \sqrt{\frac{L}{C}} = \frac{1}{\pi} \sqrt{\frac{\mu}{\epsilon}} arccosh \left( \frac{d}{2a} \right) 
\end{align}
\end{comment}

Si l es comparable o mayor que el radio del conductor $"a"$ se utilizan las expresiones para bajas frecuencias 


    $tg (\delta)$ se asume constante en f, $\sigma$ no varia linealmente con respecto a f

\begin{align}
    l \left[ \frac{dB}{m} \right] &= 20 log (e) \alpha \left[ \frac{Np}{m} \right] \\
    \alpha &= 0,2187 \frac{Np}{m} \notag
\end{align}    

\begin{align}
    Z_o &= \sqrt{ \frac{ R + jwL }  { G + jwC } }
    \label{ec 1}
\end{align}
\begin{align}
    Z_o &= 300 \Omega \   \text{( antenas receptoras de TV o FM )}   \notag
\end{align}


\begin{align}
    V_p = \frac{w}{\beta} = \frac{2 \pi f}{ \beta }
\end{align}

\begin{align}
    \beta &= \frac{ 2 \pi f } { V_p } \\
    \beta &= \frac{ 2 \pi 10Mhz } { 0,98 \cdot C } \notag \\
    \beta &= 2,1386 \frac{rad}{m} \notag
\end{align}

\begin{align*}
    v = \frac{c}{\sqrt{\epsilon_r}} = 0,986
\end{align*}

\begin{align*}
    \frac{1}{\sqrt{\epsilon_r}} = 0,98\\
    \epsilon_r = 1,0412 \  \therefore \ \text{aire}
\end{align*}

\begin{align}
    \epsilon &= \epsilon_o \cdot \epsilon_r \\
    \epsilon &= 9,2193x10^{-12} \notag
\end{align}

\begin{align*}
    \mu = \mu_0 = 1,2566x10^{-6}
\end{align*}

\begin{align}
    \gamma = \alpha + j\beta = 0,2187 + j2,1386 \\
    \gamma = \sqrt{ (R - jwL) (G + jwC) }
    \label{ec 2}
\end{align}

si $\alpha = 0 \implies \gamma = 2,0959j$
por aire $\sigma = 0$
si 
\begin{align}
G = \frac{\pi \sigma_d}{cosh^{-1}\left(\frac{d}{2a}\right) } = 0
\end{align}

Para la ecuación \ref{ec 2} y  \ref{ec 1} se busca R y L con $G = 0, Z_o = 300 \Omega, \gamma= 0,2187 + j2,1386$

\begin{align}
    \frac{d}{2a} = cosh \left (  \frac{\pi \epsilon}{C} \right) = 8,1036 ; \ \text{si} \  \frac{d} {2a} = e^{\frac{\pi \epsilon}{C}} = 8,0726
\end{align}

si conductor cobre:

\begin{align}
    l &= \sqrt{\frac{1}{\pi f \mu_o \sigma_c}} \\
    l &= 6,6085x10^{-6} = 0,066mm \  \text{;} \  l \ll a \ \text{se utiliza expresiones para alta frecuencia} \notag
\end{align}

\begin{align*}
    \mu = 0,999994 \  \text{;} \ \sigma_c = 4,8x10^7
\end{align*}

Ya que $Z_o \in \mathbb{R}$, $R \rightarrow 0$ para que esto se cumpla, por tanto:

\begin{equation}
    Z_o = \sqrt{L/C}
\end{equation}

Sabiendo que para altas frecuencias, los valores de $L$ y $C$ corresponden a las ecuaciones \eqref{eq:L} y \eqref{eq:c}. Además sustituyendo en la en la ecuación anterior.

\begin{equation} \label{eq:d}
    \begin{split}
        Z_o &= \sqrt{\frac{\frac{\mu}{\pi}ln(\frac{d}{a})}{\frac{\pi \epsilon}{ln(d/a)}}} \\
        Z_o &= \sqrt{\frac{\mu}{\epsilon}} \frac{1}{\pi} ln(\frac{d}{a}) \\
        ln(\frac{d}{a}) &= Z_o \pi \sqrt{\frac{\epsilon}{\mu}} \\
        \frac{d}{a} &= e^{Z_o \pi \sqrt{\frac{\epsilon}{\mu}}} \\
        \frac{d}{a} &= e^{Z_o \pi \sqrt{\frac{\epsilon_o}{\mu_o}}} = 12.2 > 10 \\
        d&=12.2a \\
    \end{split}
\end{equation}

Basándose en \eqref{eq:R} mientras mayor sea el diámetro del conductor mejor será nuestras aproximaciones, se considera un límite inferior donde $R=1\Omega$, por tanto

\begin{equation*}
    a>\frac{1}{\pi l \sigma_c} = 0.8mm
\end{equation*}

Se tomará un valor poco mayor, para aumentar eficiencia y disminuir costos en cobre, entonces:

$$a=1mm$$

Además utilizando \eqref{eq:d}

$$d=12.2a = 12.2mm$$

Se especificarán posteriormente los datos de la linea de transmisión diseñada.

\pagebreak
